
	\begin{UseCase}{CUC5}{Registrar ejercicio de presupuesto}{El usuario registra el ejercicio en el presupuesto.}
			\UCitem{Versión}{3.0}
			\UCitem{Estado}{Finalizado}
			\UCitem{Actor(es)}{Coordinador.}
			\UCitem{Propósito}{Que el usuario conozca los montos aprobados y ejercidos en el proyecto.}
			\UCitem{Resúmen}{El usuario realiza un ejercicio en el presupuesto, ingresando fecha y monto del ejercicio.}
			\UCitem{Entradas}{Datos de ejercicio del presupuesto. \ref{dd:DEPresupuestos}.}
			\UCitem{Salidas}{Mensaje que indica que registro de forma correcta.}
			\UCitem{Precondiciones}{Que exista al menos un presupuesto aprobado o montos por ejercer.}
			\UCitem{Postcondiciones}{Se disminuye el monto disponible del presupuesto para el proyecto en cuestión}
			\UCitem{Autor}{Hermosillo García Karen Adriana // Jessica Ramos }
			\UCitem{Referencias}{CU-P2-061011}
			\UCitem{Tipo}{Secundario. Viene del \UCref{CU2.1.2}}
			\UCitem{Módulo}{Coordinación.}
	\end{UseCase}

	\begin{UCtrayectoria}{Principal}
			\UCpaso[\UCactor] Oprime el botón \IUbutton{Ejercer} del presupuesto que se desea ejercer en la pantalla. \IUref{IURevisarAvancesProyecto}{Revisar avance de un Proyecto}.
			\UCpaso Muestra la pantalla \IUref{IUEjercerPresupuesto}{Ejercer Presupuesto} 
			\UCpaso [\UCactor] Ingresa los datos de ejercicio del presupuesto.  \label{paso:CUC5ingresarDatos}.
			\UCpaso [\UCactor] Oprime el botón \IUbutton{Aceptar}.\Trayref{A}
			\UCpaso Revisa que los datos cumplan la regla de negocio \BRref{RN2}. \Trayref{B}
			\UCpaso Revisa los datos de acuerdo al diccionario \ref{dd:DEPresupuestos} \Trayref{C}
			\UCpaso Verifica que se cumpla la regla de negocios \BRref{RN63}.
			\UCpaso Registra ejercicio del presupuesto.
			\UCpaso Muestra el mensaje de operación exitosa. \ref{MSG4}
			\UCpaso Regresa a la pantalla anterior.
	\end{UCtrayectoria}
	\newpage
	\begin{UCtrayectoriaA}{A}{Cancelar operación}{El usuario abandona el Caso de Uso.}
			\UCpaso[\UCactor] Decide no ejercer el presupuesto.
			\UCpaso[\UCactor] Oprime el botón \IUbutton{Cancelar}.
			\UCpaso Regresa a la pantalla \IUref{IURevisarAvancesProyecto}{Revisar avance de un Proyecto}.
	\end{UCtrayectoriaA}
		
	\begin{UCtrayectoriaA}{B}{Datos nulos}{Los datos ingresados por el usuario  no cumplen con la regla de negocios \BRref{2}.}
			\UCpaso Muestra el mensaje (MSG-1).\ref{MSG1}
			\UCpaso Continúa en el paso \ref{paso:CUC5ingresarDatos} del \UCref{CUC5}.
	\end{UCtrayectoriaA}
	\begin{UCtrayectoriaA}{C}{Datos incorrectos}{Los datos ingresados por el usuario  no cumplen con los datos del presupuesto \ref{dd:PresupuestoEjercido}.}
			\UCpaso Muestra el mensaje (MSG-2).\ref{MSG2}
			\UCpaso Continúa en el paso \ref{paso:CUC5ingresarDatos} del \UCref{CUC5}.
	\end{UCtrayectoriaA}

	\begin{UCtrayectoriaA}{D}{Monto invalido}{El monto debe ser mayor a 0 \BRref{1}.}
		\UCpaso Muestram el mensaje (MSG-3). \ref{MSG3}
		\UCpaso Continúa en el paso \ref{paso:CUC5ingresarDatos} del \UCref{CUC5}.
	\end{UCtrayectoriaA}

