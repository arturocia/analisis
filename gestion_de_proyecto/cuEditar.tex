%-------------------------Registrar Proyecto
	\begin{UseCase}{CUC1}{Editar datos de Proyecto}{Editar los datos de un Proyecto asociado a un Coordinador.}
		\UCitem{Versión}{1.0}
		\UCitem{Actor(es)}{Coordinador}
		\UCitem{Propósito}{Editar los datos del registro del Proyecto y establecer su Alineacion.}
		\UCitem{Resumen}{Se muestran los datos editables del Proyecto y permite asociar Ejes Tematicos y Temas Transversales y alinearlo con elementos EPN1.}
		\UCitem{Entradas}{Datos editables de Proyecto \ref{dd:DatosEditablesProyecto}.}
		\UCitem{Salidas}{Mensaje de operacion exitosa.}
		\UCitem{Precondiciones}{Se presentan las siguientes precondiciones
		  \begin{itemize}
		    \item Que existan Programas registrados.
		    \item Que existan Estructuras definidas para los Programas.
		    \item Que existan Ejes Temáticos registrados.
		    \item Que existan Temas Transversales registrados.
		    \item Que exista un Proyecto registrado.
		  \end{itemize}}
		\UCitem{Postcondiciones}{El proyecto queda listo para definir metas, indicadores, acciones y participantes.}
		\UCitem{Autor}{Adrian Martinez}
		\UCitem{Referencias}{SIDAM-BESP-P1}
		\UCitem{Tipo}{Secundario. Viene del CU XXX}
		\UCitem{Módulo}{Coordinación}	
	\end{UseCase}

	\begin{UCtrayectoria}{Principal}
			\UCpaso[\UCactor] Oprime el botón \IUbutton{Editar} de la pantalla \textit{PANTALLA DE DONDE SALE ESTA COSA!!!}.
			\UCpaso Muestra la pantalla \IUref{IUEditarDatosProyecto}{Editar Datos de Proyecto}.
			\UCpaso Obtiene los datos del Proyecto seleccionado.
			\UCpaso Obtiene los Ejes Temáticos registrados.
			\UCpaso Obtiene los Temas Transversales registrados.
			\UCpaso Obtiene los Programas registrados.
			\UCpaso Obtiene las estructuras definidas del Programa Sectorial y del Programa seleccionado.
			\UCpaso Muestra la informacion obtenida. 
			\UCpaso Deshabilita los Datos del Responsable.
			\UCpaso [\UCactor] Edita los datos del Proyecto.\label{paso:CUC1_IngresaDatosProyecto} \Trayref{A}
			\UCpaso [\UCactor] Oprime el botón \IUbutton{Aceptar}.\label{paso:CUC1_OprimeAceptar}
			\UCpaso Revisa que los datos correspondan con la Definición de Datos. \Trayref{B}
			\UCpaso Verifica que se cumpla la regla de negocio \BRref{RN44}.\Trayref{C}
			\UCpaso Verifica que se cumpla la regla de negocio \BRref{RN43}.\Trayref{D}
			\UCpaso Verifica que se cumpla la regla de negocio \BRref{RN32}.\Trayref{E}
			\UCpaso Verifica que se cumpla la regla de negocio \BRref{RN33}. \Trayref{F}
			\UCpaso Verifica que se cumpla la regla de negocio \BRref{RN40}. \Trayref{G}
			\UCpaso Valida que el periodo cumpla con  \BRref{RN60} \Trayref{H}.
			\UCpaso Valida que el periodo cumpla con  \BRref{RN7} \Trayref{H}.
			\UCpaso Registra el Proyecto.
			\UCpaso Muestra el mensaje (MSG-4) de operación exitosa.\ref{MSG4}
	\end{UCtrayectoria}

	\begin{UCtrayectoriaA}{A}{Cancelar operación}{El usuario abandona el Caso de Uso.}
			\UCpaso[\UCactor] Decide ya no editar los datos de un Proyecto.
			\UCpaso[\UCactor] Oprime el botón \IUbutton{Cancelar}.
			\UCpaso Muestra la pantalla \textit{PANTALLA DE DONDE SALE ESTA COSA}.
	\end{UCtrayectoriaA}


	\begin{UCtrayectoriaA}{B}{Datos del Proyecto Incorrectos}{Los datos de Proyecto proporcionados estan incompletos.}
			\UCpaso Muestra el mensaje (MSG-1) indicando que los datos ingresados estan incompletos. \ref{MSG1}
			\UCpaso Continúa en el paso \ref{paso:CUC1_IngresaDatosProyecto}.
	\end{UCtrayectoriaA}

	\begin{UCtrayectoriaA}{C}{Datos del proyecto incompletos}{Los datos del Proyecto no corresponden con lo especificado en el diccionario de datos.}
			\UCpaso Muestra el mensaje (MSG-2) indicando los valores que son incorrectos. \ref{MSG2}
			\UCpaso Continúa en el paso \ref{paso:CUC1_IngresaDatosProyecto}.
	\end{UCtrayectoriaA}

	\begin{UCtrayectoriaA}{D}{Alineación de Proyecto}{Proyecto que se desea registrar no esta bien alineado.}
			\UCpaso Muestra el mensaje (MSG-RN-43).\ref{MSG_RN43}
			\UCpaso Continúa en el paso \ref{paso:CUC1_IngresaDatosProyecto}.
	\end{UCtrayectoriaA}

	\begin{UCtrayectoriaA}{E}{Periodo mal definido}{El periodo del Proyecto que se desea registrar no está bien definido.}
			\UCpaso Muestra el mensaje (MSG-RN-32).\ref{MSG_RN32}
			\UCpaso Continúa en el paso \ref{paso:CUC1_IngresaDatosProyecto}.
	\end{UCtrayectoriaA}


	\begin{UCtrayectoriaA}{F}{El nombre del Proyecto repetido}{El Nombre del Proyecto ya se encuentra registrado.}
			\UCpaso Muestra el mensaje (MSG-3) \ref{MSG3}
			\UCpaso Continúa en el paso \ref{paso:CUC1_IngresaDatosProyecto}.
	\end{UCtrayectoriaA}
	\begin{UCtrayectoriaA}{G}{Las siglas del Proyecto repetidas}{Las siglas del Proyecto ya se encuentran registradas.}
			\UCpaso Muestra el mensaje (MSG-3) \ref{MSG3}
			\UCpaso Continúa en el paso \ref{paso:CUC1_IngresaDatosProyecto}.
	\end{UCtrayectoriaA}

	\begin{UCtrayectoriaA}{H}{Periodo invalido}{El periodo que se pretende dar a la nueva estructura es invalido.}
		\UCpaso Muestra el mensaje (MSG-RN-7).\ref{MSG_RN7}.
		\UCpaso Continua en el paso \ref{paso:CUC1_IngresarDatosProyecto}.
	\end{UCtrayectoriaA}




%-------------------------------------- TERMINA descripción del caso de uso.