\subsection{Pantalla: Consultar bitacora coordinador}

\subsubsection{Objetivo}
  Mostrar los mensajes acerca de restricciones del proyecto de o hacia el coordinador y permitir registrar restricciones al proyecto.

\subsubsection{Diseño}
\IUfig[0.8]{CUC2/consultaBitacoraCoordinador.png}{IUConsultaBitacoraCoordinador}{Consulta bitacora, vista del coordinador.}
\IUfig[0.8]{CUC2/consutlaBitacoraCoordinador_registraRestriccion.png}{IURegistraRestriccionBitacoraCoordinador}{Registra restricción en la bitacora, vista del coordinador.}

\subsubsection{Salidas}
  En esta pantalla se muestran los mensajes entre el coordinador con los demas usuarios agrupados por asunto y estan ordenados de forma descendente con respecto a la fecha de los asuntos.
  Para la vista se usara un acordeon para expandir/contrare los mensajes de cada asunto.

  Tipos de mensajes
  
  Describir orden: Primero las gestiones y luego los avisos. Las gestiones se ordenan por estatus, primero las turnadas, luego las que estan pendientes y por ultimas las que estan atendidas. Al interior cada una se ordenara por fecha de ultima atencion descendente.

  Los avisos se ordenan por fecha en orden descendente.

\subsubsection{Controles}
\begin{itemize}
 \item Expandir y contraer mensajes: Esta opción permite al coordinador ver a detalle los datos de la restricción u ocultarlos para poder ver otras restricciones.
\end{itemize}


\subsubsection{Comandos}
\begin{itemize}
 \item \IUbutton{Registrar Restricción}: El sistema actualizara la pantalla \IUref{IUConsultaBitacoraCoordinador}{Consulta bitacora, vista del coordinador.} para poder registrar los datos de la restricción.
 \item \IUbutton{Registrar} El sistema actualizara la pantalla \IUref{IURegistraRestriccionBitacoraCoordinador}{Registra restricción en la bitacora, vista del coordinador.} para poder registrar la restricción siempre que los campos esten completos y mostrará el MSG4. En otro caso el sistema mostrará el MSG2 (por la regla \BRref{RN1})). 
 \item \IUbutton{Cancelar}: El sistema actualizara la pantalla \IUref{IURegistraRestriccionBitacoraCoordinador}{Registra restricción en la bitacora, vista del coordinador.} para eliminar los campos de registrar restriccion y no registrarla.
\end{itemize}

\subsubsection{Mensajes}
\begin{itemize}
  \item MSG2: Debe ingresar todos los datos.
  \item MSG4: El registro se ha realizado exitosamente.
\end{itemize}