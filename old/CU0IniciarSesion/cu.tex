  	\begin{UseCase}{CU0}{Iniciar Sesión}{Permite acceder al sistema.}
		\UCitem{Versión}{1.0}
		\UCitem{Actor(es)}{Administrador, Coordinador, Gerente y Director}
		\UCitem{Propósito}{Restringir el acceso al sistema a solo personal autorizado.}
		\UCitem{Resumen}{Autentica los usuarios registrados.}
		\UCitem{Entradas}{Nombre de login y Contraseña\ref{dd:Usuario}.}
		\UCitem{Salidas}{Menú de la Sesión iniciada.}
		\UCitem{Precondiciones}{Ninguna.}
		\UCitem{Postcondiciones}{El actor inicia sesión.}
		\UCitem{Autor}{Vargas Chávez Alejandro.}
		\UCitem{Referencias}{SIDAM-BESP-P0-Especificación de Catálogos}
		\UCitem{Tipo}{Primario.}
		\UCitem{Módulo}{Control de acceso}
	\end{UseCase}	

	\begin{UCtrayectoria}{Principal}
		\UCpaso [\UCactor] Ingresa al Sistema.
		\UCpaso Muestra la pantalla \IUref{IULoggueo}{Iniciar Sesión}. 
		\UCpaso [\UCactor] Ingresa el login y contraseña.\label{paso:CU0ingresarDatos}
		\UCpaso [\UCactor] Oprime el botón \IUbutton{Aceptar}.
		\UCpaso Valida los datos ingresados.\Trayref{A}
		\UCpaso Obtiene el perfil del usuario.
		\UCpaso Muestra el menú correspondiente al perfil obtenido.
	\end{UCtrayectoria}

	\begin{UCtrayectoriaA}{A}{Datos Incorrectos}{Los Datos ingresados no son correctos.}
			\UCpaso Muestra el mensaje (MSG2-2) indicando que los datos ingresados no son correctos.\ref{MSG2-2}
			\UCpaso Continúa con el paso \ref{paso:CU0ingresarDatos}.
		\end{UCtrayectoriaA}

	\subsection{Puntos de Extensión}
	\UCEPdef{Recuperar Contraseña}
	{El actor ha olvidado su contraseña.}
	{\ref{paso:CU0ingresarDatos}}
	{\UCref{CU0}}

	\begin{UseCase}{CU0.1}{Recuperar Contraseña}{El actor decide recuperar su contraseña.}
			\UCitem{Versión}{1.0}
			\UCitem{Actor(es)}{Administrador, Coordinador, Gerente y Director}
			\UCitem{Propósito}{Recuperar Contraseña.}
			\UCitem{Resumen}{El sistema manda un correo electrónico a la dirección de correo principal proporcionando el usuario y password del usuario ingresado.}
			\UCitem{Entradas}{Login de Usuario.}
			\UCitem{Salidas}{Mensaje de confirmación.}
			\UCitem{Precondiciones}{Que el usuario tenga acceso al sistema.}
			\UCitem{Postcondiciones}{Correo electrónico enviado.}
			\UCitem{Autor}{Vargas Chávez Alejandro.}
			\UCitem{Referencias}{SIDAM-BESP-P0-Especificación de Catálogos}
			\UCitem{Tipo}{Secundario. Viene del caso de uso \UCref{CU0}}
			\UCitem{Módulo}{Control de acceso}
		\end{UseCase}

	\begin{UCtrayectoria}{Principal}
			\UCpaso[\UCactor] Oprime el vínculo \underline{¿Ha olvidado su contraseña?} en la página \IUref{IULoggueo}{Iniciar Sesión}.
			\UCpaso Muestra la pantalla \IUref{IURecuperarPassword}{Recuperar Contraseña}	
			\UCpaso [\UCactor] Introduce el usuario.\Trayref{A}\label{paso:CU0.1recpass}
			\UCpaso [\UCactor] Oprime el botón \IUbutton{Aceptar} 
			\UCpaso Busca los datos asociados al usuario ingresado.\Trayref{B}
			\UCpaso Envia un correo electrónico al e-mail asociado al usuario ingresado.\Trayref{C}
			\UCpaso Muestra un mensaje indicando que se ha enviado un mensaje a su correo electrónico.
	\end{UCtrayectoria}

		\begin{UCtrayectoriaA}{A}{Cancelar operación}{El actor decide ya recuperar su contraseña}
			\UCpaso[\UCactor] Oprime el botón \IUbutton{Regresar}.
			\UCpaso Continúa en el paso \ref{paso:CU0ingresarDatos} del \UCref{CU0}.
		\end{UCtrayectoriaA}

		\begin{UCtrayectoriaA}{B}{Datos Incorrectos}{El usuario ingresado no existe.}
			\UCpaso Muestra el mensaje (MSG2-3) indicando que el usuario ingresado no existe.\ref{MSG2-3}
			\UCpaso Continúa en el paso \ref{paso:CU0.1recpass}.
		\end{UCtrayectoriaA}

		\begin{UCtrayectoriaA}{C}{E-mail no valido}{La cuenta de correo electrónico asociado al usuario ingresado no es correcto.}
			\UCpaso Muestra un mensaje indicando que ha ocurrido un error con la cuenta de correo electrónico.
			\UCpaso Continúa en el paso \ref{paso:CU0.1recpass}.
		\end{UCtrayectoriaA}