\subsection{Pantalla: Modificar Eje Temático}

\subsubsection{Objetivo}
Dar la posibilidad de realizar cambios en las áreas. Vease Figura ~\ref{IUModificarEjeTematico}, viene de Gestionar Ejes Temáticos.

\IUfig[0.6]{CUG3/ModificarEjeTematico.PNG}{IUModificarEjeTematico}{Modificar Eje Temático.}

\subsubsection{Salidas}
Datos de eje temático actuales \ref{dd:EjeTematico}.
\begin{itemize}
\item Nombre del eje temático
\item Descripción del eje temático
\end{itemize}
Mostrará el MSG4 \ref{MSG4}

\subsubsection{Entradas}
Descripción del eje temático a modificar.
\subsubsection{Controles}
\begin{itemize}
\item Frente al \textit{label} Descripción se encuentra un \textit{textbox} para que el usuario modifique la descripción del eje temático.
\end{itemize}
\subsubsection{Comandos}
\begin{itemize}
 \item \IUbutton{Aceptar} Actualiza los datos del eje temático que el usuario ha ingresado siempre y cuando sean correctos. Cuando los datos no cumplan con la regla de negocios  \BRref{2} mostrará el MSG1 \ref{MSG1}, si no cumple con el diccionario de datos \ref{dd:EjeTematico} mostrará el MSG2 \ref{MSG2}. Cuando se modifique el Eje temático de forma correcta mostrará el MSG3 \ref{MSG3} 
 \item \IUbutton{Cancelar}: Al presionar este botón cancela el proceso y regresa a la página \IUref{IUGestEjesTematicos}{Gestionar Ejes Temáticos}.

\end{itemize}


