\subsection{Pantalla: Modificar Área}

\subsubsection{Objetivo}
Dar la posibilidad de realizar cambios en las áreas. Vease Figura ~\ref{IUModificarArea}, viene de Gestionar Áreas.

\IUfig[0.6]{CU7/ModificarArea.PNG}{IUModificarArea}{Modificar Área.}

\subsubsection{Salidas}
Datos de Área sin modificar \ref{dd:Area}.
\begin{itemize}
 \item Nombre del área
 \item Siglas del área
 \item Descripción del área
\end{itemize}
\subsubsection{Entradas}
Descripción de área.


\subsubsection{Comandos}
\begin{itemize}
 \item \IUbutton{Aceptar} Actualiza los datos de el área que el usuario ha ingresado siempre y cuando sean correctos. Cuando los datos no cumplan con la regla de negocios  \BRref{2} mostrará el MSG1 \ref{MSG1}, o el diccionario de datos \ref{dd:Area} mostrará el MSG2 \ref{MSG2}. 
 \item \IUbutton{Cancelar}: Al presionar este botón cancela el proceso y regresa a la página \IUref{IUGestAreas}{Gestionar Áreas}.

\end{itemize}

