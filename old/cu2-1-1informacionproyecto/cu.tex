% Descripción: Describe la funcionalidad ofrecida por el CU
% Propósito: Describe el objetivo o razón de ser del CU
% Resumen: Describe brevemente lo que hace el CU

	\begin{UseCase}{CU2.1.1}{Revisar Información de un Proyecto}{Ofrece un mecanismo para ver los datos de un proyecto registrado y tiene la opción de editar un proyecto.}
		\UCitem{Versión}{3.0}
		\UCitem{Estado}{Finalizado}
		\UCitem{Actor(es)}{Coordinador}
		\UCitem{Propósito}{Contar con un mecanismo que ayude a revisar los datos de un proyecto.}
		\UCitem{Resumen}{Se muestran los datos de un proyecto registrado con la posibilidad de modificar.}
		\UCitem{Entradas}{Ninguna.}
		\UCitem{Salidas}{Datos del proyecto seleccionado.}
		\UCitem{Precondiciones}{Que exista al menos un proyecto registrado.}
		\UCitem{Postcondiciones}{Ninguna.}
		\UCitem{Autor}{Hermosillo García Karen Adriana }
		\UCitem{Referencias}{CU-P2-061011}
		\UCitem{Tipo}{Secundario.}
		\UCitem{Módulo}{Coordinación.}
	\end{UseCase}
	
	% 1.- escriba solo una trayectoria principal
	% 2.- El actor es quien siempre inicia un CU
	% 3.- evite usar ``si ... entonces ...'' o ``mientras ...'' o ``para cada ...''
	% 4.- No olvide mencionar todas las verificaciones y cálculos realizados por el sistema
	% 5.- Evite mencionar palabras como: Tabla, BD, conexión, etc.
	
	
	
	
	\begin{UCtrayectoria}{Principal}
		\UCpaso[\UCactor] Selecciona la opción \IUbutton{Revisar Información de Proyecto} de la pantalla \IUref{IURevisarProyecto}{Menú Revisar Proyecto}.
		\UCpaso Muestra en la pantalla \IUref{IURevisarInformacionProyecto}{Revisar Información del Proyecto} los datos bloqueados del proyecto seleccionado.\UCExtensionPoint{CUC1}{Editar Proyecto}
		\UCpaso [\UCactor] Oprime el botón \IUbutton{Aceptar}.
		\UCpaso Continua con el paso \ref{paso:CU2.1RevisarProyecto}. 
	\end{UCtrayectoria}
%-------------------------------------- TERMINA descripción del caso de uso.