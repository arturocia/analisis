% Descripción: Describe la funcionalidad ofrecida por el CU
% Propósito: Describe el objetivo o razón de ser del CU
% Resumen: Describe brevemente lo que hace el CU
%  
  
%========Aqui se describen los casos de uso que se derivan de el caso de uso CU2.1.3 Revisar seguimiento de un proyecto

%--------- Atender restricción del proyecto
	\begin{UseCase}{CUG1.1.4.1}{Atender restricción.}{Permite enviar instrucciones al coordinador, en respuesta a una restricción.}
		\UCitem{Versión}{1}
		\UCitem{Actor(es)}{Gerente.}
		\UCitem{Propósito}{El gerente pueda comunicar los pasos se deben realizar para dar solución a la restricción.}
		\UCitem{Resumen}{El gerente evalua las restricciones en el proyecto, decide que pasos se deben realizar y los comunica por medio de un mensaje.}
		\UCitem{Entradas}{Pasos a seguir para la solución de la restriccion.\ref{dd:Atencion}}
		\UCitem{Salidas}{Mensaje de confirmación.\ref{MSG5}}
		\UCitem{Precondiciones}{El proyecto debe estar en ejecución.}
		\UCitem{Postcondiciones}{Se registra la respuesta a la restriccion.}
		\UCitem{Autor}{Torres Govea Miguel Angel}
		\UCitem{Referencias}{SIDAM-BESP-P1-Especificación de Catálogos}
		\UCitem{Tipo}{Secundario. Viene del \UCref{CU2.1.3}.}
		\UCitem{Módulo}{Gerencia.}
	\end{UseCase}
		
      
	\begin{UCtrayectoria}{Principal}
		\UCpaso[\UCactor] Da clic en el boton \IUbutton{Atender restricción.} de una restricción por atender o en una atención enviada por el secretario \IUref{IUConsultaBitacoraGerente}{Consulta a la bitacora del proyecto, vista
del gerente.}.
		\UCpaso Actualiza la pantalla para la atencion de la restriccion.\IUref{IUConsultaBitacoraGerenteAtenderRestriccion}{Consulta a la bitacora del proyecto, vista
del gerente, atender restricción.}
		\UCpaso [\UCactor] Agrega las observaciones de la restricción y selecciona la decisión de desbloqueo del proyecto.\label{paso:CUG3ingresaDatosTurnar}
		\UCpaso [\UCactor] Da clic en el boton \IUbutton{Atender Restricción}. 
		\UCpaso Verifica que se cumpla la regla de negocio \BRref{RN2}. \Trayref{B}
		\UCpaso Verifica que el registro coincida con la definicion en el diccionario de datos \ref{dd:Atencion}. \Trayref{B}
		\UCpaso Registra la atención con referencia a la restricción o atencion original.
		\UCpaso El sistema actualiza el estatus del proyecto en funcion de a decisión de desbloqueo: ``Si'' pasa a  ``Edicion'', ``No'' se queda en ``Ejecuación'' .
		\UCpaso Muestra el mensaje (MSG-4) de operación exitosa.\ref{MSG4}
	\end{UCtrayectoria}
		
	\begin{UCtrayectoriaA}{A}{Cancelar operación}{El usuario abandona el Caso de Uso.}
			\UCpaso[\UCactor] Decide ya no registrar la atención.
			\UCpaso[\UCactor] Oprime el botón \IUbutton{Cancelar}.
			\UCpaso Actualiza la pantalla removiendo el campo para turnar restricción y no la registra. \IUref{IUConsultaBitacoraGerente}{Consulta a la bitacora del proyecto, vista
del secretario.}
	\end{UCtrayectoriaA}

	\begin{UCtrayectoriaA}{B}{Datos de la Atención Incompletos.}{Algunos campos del formulario no han sido capturados.}
			\UCpaso Muestra el mensaje (MSG-1) indicando al usuario verifique que todos los campos hayan sido capturados.\ref{MSG1}
			\UCpaso Continúa en el paso \ref{paso:CUG3ingresaDatosTurnar} del \UCref{CUS1.4}.
	\end{UCtrayectoriaA}

	\begin{UCtrayectoriaA}{C}{Datos de la Atención Inconsistentes.}{Algunos datos no son validos.}
			\UCpaso Muestra el mensaje (MSG-1) indicando al usuario verifique que todos los campos son validos.\ref{MSG1}
			\UCpaso Continúa en el paso \ref{paso:CUG3ingresaDatosTurnar} del \UCref{CUS1.4}.
	\end{UCtrayectoriaA}