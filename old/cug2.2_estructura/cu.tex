
	\begin{UseCase}{CUG2.2}{Gestionar Estructura}{Administra las estructuras con las opciones de agregar, modificar y eliminar.}
		\UCitem{Versión}{1.0}
		\UCitem{Actor(es)}{Gerente.}
		\UCitem{Propósito}{Contar con un mecanismo sencillo que ayude a definir la estructura de los programas de primer nivel (EP1N) registrados.}
		\UCitem{Resumen}{Se muestra la estructura del programa asociado al Gerente con su diagrama de gantt con la 
		posibilidad de agregar, modificar y eliminar elementos con base en los niveles definidos.}
		\UCitem{Entradas}{No.}
		\UCitem{Salidas}{Estructura y diagrama de gantt.}
		\UCitem{Precondiciones}{
			\begin{itemize}
				\item Haber pasado por el CU0 y obtenido el perfil de Gerente.
				\item Que exista al menos programa asociado al usuario y que los niveles de este último esten definidos.
			\end{itemize}
		 }
		\UCitem{Postcondiciones}{Ninguna.}
		\UCitem{Autor}{Ernesto Alvarado.}
		\UCitem{Referencias}{}
		\UCitem{Tipo}{Primario.}
		\UCitem{Módulo}{Gerencia}
	\end{UseCase}

	\begin{UCtrayectoria}{Principal}
		\UCpaso[\UCactor] Selecciona la opción \IUbutton{Gestionar Estructuras} en el menú \IUref{IUMenuGerente}{Menú de gerente}.
		\UCpaso Muestra la pantalla \IUref{IUEstructura}{Gestión de Estructura}
		\UCpaso Busca el programas asociado al usuario.
		\UCpaso Construye la gráfica de gantt con los elementos de primer nivel del programa. \Trayref{A} \Trayref{B}  \label{paso:CUG2.2gantt}
		\UCpaso Presiona salir.
	\end{UCtrayectoria}

	\begin{UCtrayectoriaA}{A}{Expandir elemento.}{El usuario puede expandir cada elemento del diagrama que tenga elementos hijos asociados.}
			\UCpaso[\UCactor] Solicita expandir un elemento.
			\UCpaso Busca los elementos hijos del elemento seleccionado para expandir.
			\UCpaso Actualiza el diagrama de gantt con los elementos hijos del elemento seleccionado.
			\UCpaso Regresa al paso \ref{paso:CUG2.2gantt} de \UCref{CUG2.2}.
	\end{UCtrayectoriaA}

	\begin{UCtrayectoriaA}{B}{Mostrar datos del elemento.}{El usuario puede seleccionar un elemento del diagrama de gantt para que el sistema muestre los datos del mismo.}
			\UCpaso[\UCactor] Solicita ver los datos un elemento.
			\UCpaso Busca el elemento seleccionado.
			\UCpaso Muestra los datos del elemento seleccionado. Puntos de extension 1:\UCref{CUG2.2.1} 2:\UCref{CUG2.2.2} 3:\UCref{CUG2.2.3}.
			\UCpaso Regresa al paso \ref{paso:CUG2.2gantt} de \UCref{CUG2.2}. 
	\end{UCtrayectoriaA}
	  
	\begin{UseCase}{CUG2.2.1}{Registrar elemento estructura}{Permite al usuario registrar un nuevo elemento estructura.}
		\UCitem{Versión}{1.0}
		\UCitem{Actor(es)}{Gerente.}
		\UCitem{Propósito}{Agregar una nueva estructura.}
		\UCitem{Resumen}{Se agrega una estructura registrando sus datos.}
		\UCitem{Entradas}{Datos de la estructura \ref{dd:DatosEstructura}.}
		\UCitem{Salidas}{Diagrama de gantt con la nueva estructura incluida.}
		\UCitem{Precondiciones}{Haber pasado por el \UCref{CUG2.2} trayectoria \Trayref{B} .}
		\UCitem{Postcondiciones}{Se registra la nueva estructura.}
		\UCitem{Autor}{Ernesto Alvarado Gaspar.}
		\UCitem{Referencias}{SIDAM-BESP-P1}
		\UCitem{Tipo}{Secundario. Viene del \UCref{CUG2.2}}
		\UCitem{Módulo}{Gerencia}
	\end{UseCase}

	\begin{UCtrayectoria}{Principal}
		\UCpaso[\UCactor] Solicita registrar una nueva estructura.
		\UCpaso Valida que se cumpla la  \BRref{RN48} \Trayref{F}.
		\UCpaso Define como padre del nuevo elemento el seleccionado en el \UCref{CUG2.2} trayectoria \Trayref{B}. 
		\UCpaso Define el nivel del nuevo elemento con base en el nivel del seleccionado.
		\UCpaso Calcula el periodo definido o relativo que limita al del elemento nuevo.
		\UCpaso Calcula el camino de migajas del elemento seleccionado.
		\UCpaso Muestra la pantalla \IUref{IUAgregarEstructura}{Agregar estructura}
		\UCpaso [\UCactor] Ingresa los datos de la estructura \ref{dd:DatosEstructura}. \ref{paso:CUG2.2.2editaEstructura}
		\UCpaso [\UCactor] Solicita registrar el elemento. \Trayref{A} \Trayref{B}
		\UCpaso Valida que el periodo cumpla con  \BRref{RN33} \Trayref{C}.
		\UCpaso Valida la  \BRref{RN32} \Trayref{D}.
		\UCpaso Valida que el periodo cumpla con  \BRref{RN7} \Trayref{E}.
		\UCpaso Valida que el periodo cumpla con  \BRref{RN60} \Trayref{E}.
		\UCpaso Registra la nueva estructura.
		\UCpaso Regresa al paso \ref{paso:CUG2.2gantt}.
		\UCpaso Muestra un mensaje (MSG-4) de registro exitoso.\ref{MSG4}
	\end{UCtrayectoria}

	\begin{UCtrayectoriaA}{A}{Cancelar}{No desea modificar la estructura seleccionada.}
		\UCpaso[\UCactor] Indica que no desea continuar con el CU.
		\UCpaso Continua en el paso \ref{paso:CUG2.2gantt}.
	\end{UCtrayectoriaA}

	\begin{UCtrayectoriaA}{B}{Datos erroneos}{Introduce datos erroneos.}
		\UCpaso Muestra los campos que no estan debidamente llenados según el diccionario de datos \ref{dd:DatosEstructura}.
		\UCpaso Continua en el paso \ref{paso:CUG2.2.2editaEstructura}.
	\end{UCtrayectoriaA}

	\begin{UCtrayectoriaA}{C}{Nombre repetido}{El nombre que se pretende dar a la nueva estructura ya esta siendo usado.}
		\UCpaso Muestra el mensaje (MSG-3).\ref{MSG3}.
		\UCpaso Continua en el paso \ref{paso:CUG2.2.2editaEstructura}.
	\end{UCtrayectoriaA}

	\begin{UCtrayectoriaA}{D}{Periodo mal definido}{El periodo que se pretende dar a la nueva estructura no esta bien definido.}
		\UCpaso Muestra el mensaje (MSG-RN-32).\ref{MSG_RN32}.
		\UCpaso Continua en el paso \ref{paso:CUG2.2.2editaEstructura}.
	\end{UCtrayectoriaA}

	\begin{UCtrayectoriaA}{E}{Periodo invalido}{El periodo que se pretende dar a la nueva estructura es invalido.}
		\UCpaso Muestra el mensaje (MSG-RN-7).\ref{MSG_RN7}.
		\UCpaso Continua en el paso \ref{paso:CUG2.2.2editaEstructura}.
	\end{UCtrayectoriaA}

	\begin{UCtrayectoriaA}{E}{Nivel de estructura no válido}{La estructura que desea agregar supera el maximo nivel definido para el programa.}
		\UCpaso Regresa al paso 1 del \UCref{CUG2.2}.
		\UCpaso Muestra el mensaje (MSG-RN-48).\ref{MSG_RN48}.
	\end{UCtrayectoriaA}

%-------------------------------------------------------
	\begin{UseCase}{CUG2.2.2}{Modificar elemento estructura}{El usuario selecciona una estructura para modificar sus datos.}
			\UCitem{Versión}{1.0}
			\UCitem{Actor(es)}{Gerente.}
			\UCitem{Propósito}{Modificar los datos de una estructura y actualizar su registro en el sistema.}
			\UCitem{Resumen}{El sistema muestra los datos de la estructura seleccionada para la modificación de sus datos.}
			\UCitem{Entradas}{Nuevos datos para la estructura seleccionada \ref{dd:DatosEditablesEstructura}.}
			\UCitem{Salidas}{Diagrama gantt con la estructura modificadas.}
			\UCitem{Precondiciones}{Haber pasado por el \UCref{CUG2.2} trayectoria \Trayref{B}.}
			\UCitem{Postcondiciones}{Se actualizan los datos de la estructura seleccionada.}
			\UCitem{Autor}{Ernesto Alvarado Gaspar.}
			\UCitem{Referencias}{SIDAM-BESP-P1}
			\UCitem{Tipo}{Secundario. Viene del \UCref{CUG2.2}}
			\UCitem{Módulo}{Gerencia}
	\end{UseCase}

	\begin{UCtrayectoria}{Principal}
		\UCpaso[\UCactor] Solicita modificar la estructura seleccionada.
		\UCpaso Muestra los datos editables de la estructura para su edición como se definen en \ref{dd:DatosEditablesEstructura}. \label{paso:CUG2.2.2editaEstructura}
		\UCpaso Calcula el periodo definido o relativo que limita al del elemento nuevo.
		\UCpaso Calcula el camino de migajas del elemento seleccionado.
		\UCpaso Muestra la pantalla \IUref{IUModificarEstructura}{Modificar estructura}
		\UCpaso [\UCactor] Modifica los datos de la estructura.
		\UCpaso [\UCactor] Solicita registrar los cambios. \Trayref{A}
		\UCpaso Valida la \BRref{RN33} \Trayref{B}.
		\UCpaso Valida la \BRref{RN32} \Trayref{D}. \Trayref{D}
		\UCpaso Verifica que los periodos de este elemento y sus hijos cumplan con \BRref{RN7} \Trayref{C}.
		\UCpaso Verifica que los periodos de este elemento y sus hijos cumplan con \BRref{RN60} \Trayref{C}.
		\UCpaso Registra los cambios a los datos de la estructura.
		\UCpaso Regresa al paso \ref{paso:CUG2.2gantt}.
		\UCpaso Muestra un mensaje (MSG-4) de aviso de edición.\ref{MSG4}
	\end{UCtrayectoria}

	\begin{UCtrayectoriaA}{A}{Cancelar}{No desea modificar la estructura seleccionada.}
		\UCpaso[\UCactor] Indica que no desea continuar con el CU.
		\UCpaso Continua en el paso \ref{paso:CUG2.2gantt}.
	\end{UCtrayectoriaA}

	\begin{UCtrayectoriaA}{B}{Datos erroneos}{Introduce datos erroneos.}
		\UCpaso Muestra los campos que no estan debidamente llenados según el diccionario de datos \ref{dd:DatosEditablesEstructura}.
		\UCpaso Continua en el paso \ref{paso:CUG2.2.2editaEstructura}.
	\end{UCtrayectoriaA}

	\begin{UCtrayectoriaA}{C}{Periodo invalido}{El periodo de la nueva estructura o de alguno de sus hijos es invalido.}
		\UCpaso Muestra el mensaje (MSG-RN-7)\ref{MSG_RN7}.
		\UCpaso Continua en el paso \ref{paso:CUG2.2.2editaEstructura}.
	\end{UCtrayectoriaA}

	\begin{UCtrayectoriaA}{D}{Periodo mal definido}{El periodo que se pretende dar a la nueva estructura no esta bien definido.}
		\UCpaso Muestra el mensaje (MSG-RN-32).\ref{MSG_RN32}.
		\UCpaso Continua en el paso \ref{paso:CUG2.2.2editaEstructura}.
	\end{UCtrayectoriaA}


%--------- Eliminar Unidad
	\begin{UseCase}{CUG2.2.3}{Eliminar estructura}{El usuario elimina una estructura registrada.}
			\UCitem{Versión}{1.0}
			\UCitem{Actor(es)}{Gerente.}
			\UCitem{Propósito}{Eliminar la estructura seleccionada.}
			\UCitem{Resumen}{El sistema muestra los datos de la estructura seleccionada para confirmar su eliminación.}
			\UCitem{Entradas}{Identificador de la Estructura seleccionada.}
			\UCitem{Salidas}{Datos de la estructura seleccionada \ref{dd:DatosEstructura}.}
			\UCitem{Precondiciones}{Haber pasado por el \UCref{CUG2.2} trayectoria \Trayref{B}.}
			\UCitem{Postcondiciones}{La estructura se elimina junto con sus descendientes de los registros.}
			\UCitem{Autor}{Ernesto Alvarado Gaspar.}
			\UCitem{Referencias}{SIDAM-BESP-P1}
			\UCitem{Tipo}{Secundario. Viene del \UCref{CUG2.2}}
			\UCitem{Módulo}{Gerencia}
	\end{UseCase}

	\begin{UCtrayectoria}{Principal}
		\UCpaso[\UCactor] Solicita eliminar la estructura seleccionada.
		\UCpaso Muestra los datos de la estructura seleccionada. \label{paso:CUG2.2.3eliminarEstructura}
		\UCpaso [\UCactor] Solicita eliminar la estructura. \Trayref{A}
		\UCpaso Valida la \BRref{RN36}. \Trayref{B}
		\UCpaso La estructura seleccionada se elimina.
		\UCpaso Elimina la estructura y todos sus descendientes.
		\UCpaso Regresa al paso \ref{paso:CUG2.2gantt}.
		\UCpaso Muestra un mensaje (MSG-4) de aviso de eliminación.\ref{MSG4}
	\end{UCtrayectoria}

	\begin{UCtrayectoriaA}{A}{Cancelar}{No desea eliminar una nueva estructura.}
		\UCpaso[\UCactor] Indica que no desea continuar con el CU.
		\UCpaso Continua en el paso \ref{paso:CUG2.2gantt}.
	\end{UCtrayectoriaA}

	\begin{UCtrayectoriaA}{B}{Estructura ya alineada}{La estructura seleccionada ya tiene proyectos alineados a ella.}
		\UCpaso Muestra el mensaje (MSG-RN-36).\ref{MSG_RN36}.
		\UCpaso Continua en el paso \ref{paso:CUG2.2.3eliminarEstructura}.
	\end{UCtrayectoriaA}