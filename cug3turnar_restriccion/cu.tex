% Descripción: Describe la funcionalidad ofrecida por el CU
% Propósito: Describe el objetivo o razón de ser del CU
% Resumen: Describe brevemente lo que hace el CU
%  

%--------- Turnar restricción del proyecto
	\begin{UseCase}{CUG3}{Turnar restricción.}{Envía una descripción al secretario para solicitar instrucciones sobre como atender la misma.}
		\UCitem{Versión}{1}
		\UCitem{Actor(es)}{Gerente.}
		\UCitem{Propósito}{El secretario este enterado de un asunto que requiere atencion especial en el proyecto.}
		\UCitem{Resumen}{El gerente al evaluar las restricciones del proyecto decidira si se debe turnar el proyecto.}
		\UCitem{Entradas}{Datos de la restricción.\ref{dd:Atencion}}
		\UCitem{Salidas}{Mensaje de confirmación.\ref{MSG5}}
		\UCitem{Precondiciones}{El proyecto debe estar en ejecución.}
		\UCitem{Postcondiciones}{Se registra la peticion para la atención de la restricción en estado de pendiente.}
		\UCitem{Autor}{Torres Govea Miguel Angel}
		\UCitem{Referencias}{SIDAM-BESP-P1-Especificación de Catálogos}
		\UCitem{Tipo}{Secundario. Viene del \UCref{CU2.1.3}.}
		\UCitem{Módulo}{Gerencia.}
	\end{UseCase}
		
	\begin{UCtrayectoria}{Principal}
		\UCpaso[\UCactor] Da clic en el boton \IUbutton{Turnar restricción.} en la restricción a turnar \IUref{IUConsultaBitacoraGerente}{Consulta a la bitacora del proyecto, vista
del gerente.}.
		\UCpaso Actualiza la pantalla para la atencion de la restriccion.\IUref{IUConsultaBitacoraGerenteTurnarRestriccion}{Consulta a la bitacora del proyecto, vista
del gerente, atender restricción.}
		\UCpaso [\UCactor] Agrega las instrucciones para la solución de la restricción.\label{paso:CUG1.4.1ingresaDatosAtencion}
		\UCpaso [\UCactor] Da clic en el boton \IUbutton{Turnar Restricción}. 
		\UCpaso Verifica que se cumpla la regla de negocio \BRref{RN2}. \Trayref{B}
		\UCpaso Verifica que el registro coincida con la definicion en el diccionario de datos \ref{dd:Atencion}. \Trayref{B}
		\UCpaso Registra la atención.
		\UCpaso Muestra el mensaje (MSG-4) de operación exitosa.\ref{MSG4}
	\end{UCtrayectoria}
		
	\begin{UCtrayectoriaA}{A}{Cancelar operación}{El usuario abandona el Caso de Uso.}
			\UCpaso[\UCactor] Decide ya no registrar la atención.
			\UCpaso[\UCactor] Oprime el botón \IUbutton{Cancelar}.
			\UCpaso Actualiza la pantalla removiendo el campo de atención y no la registra. \IUref{IUConsultaBitacoraGerente}{Consulta a la bitacora del proyecto, vista
del secretario.}
	\end{UCtrayectoriaA}

	\begin{UCtrayectoriaA}{B}{Datos de la Atención Incompletos.}{Algunos campos del formulario no han sido capturados.}
			\UCpaso Muestra el mensaje (MSG-1) indicando al usuario verifique que todos los campos hayan sido capturados.\ref{MSG1}
			\UCpaso Continúa en el paso \ref{paso:CUG1.4.1ingresaDatosAtencion} del \UCref{CUS1.4}.
	\end{UCtrayectoriaA}

	\begin{UCtrayectoriaA}{C}{Datos de la Atención Inconsistentes.}{Algunos datos no son validos.}
			\UCpaso Muestra el mensaje (MSG-1) indicando al usuario verifique que todos los campos son validos.\ref{MSG1}
			\UCpaso Continúa en el paso \ref{paso:CUG1.4.1ingresaDatosAtencion} del \UCref{CUS1.4}.
	\end{UCtrayectoriaA}