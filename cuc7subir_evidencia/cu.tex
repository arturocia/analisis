
%--------- Atender restricción en el proyecto
	\begin{UseCase}{CUC7}{Subir evidencia.}{Permite subir una nueva evidencia a la acción seleccionada.}
		\UCitem{Versión}{1}
		\UCitem{Actor(es)}{Coordinador.}
		\UCitem{Propósito}{Agregar una evidencia a la acción de un proyecto.}
		\UCitem{Resumen}{El Director Gerete podrá agregar una evidencia al proyecto.}
 		\UCitem{Entradas}{Datos de la evidencia \ref{dd:Evidencia}.}
		\UCitem{Salidas}{Ninguna}
		\UCitem{Precondiciones}{Deberá estar seleccionada una acción.}
		\UCitem{Postcondiciones}{Se agrega una evidencia a la acción del proyecto.}
		\UCitem{Autor}{Itzel Medrano Carrasco}
		\UCitem{Referencias}{SIDAM-BESP-P1-Especificación de Catálogos}
		\UCitem{Tipo}{Secundario. Viene del \UCref{CU2.1.2.1}.}
		\UCitem{Módulo}{Coordinación.}
	\end{UseCase}
		
	\begin{UCtrayectoria}{Principal}
		\UCpaso[\UCactor] Da clic en la opción \IUbutton{Agregar Evidencia} en la pantalla \IUref{IURevisarReportesAvance}{Revisar Reportes de Avance}.
		\UCpaso Muestra la pantalla \IUref{IUSubirEvidencia}{Subir Evidencia}
		\UCpaso [\UCactor] Ingresa los datos de la evidencia \ref{dd:Evidencia}.
		\UCpaso [\UCactor] Da clic en la opción \IUbutton{Agregar}\Trayref{A}
		\UCpaso Verifica que se cumpla la regla de negocio \BRref{RN2}. \Trayref{D}
		\UCpaso Verifica que los datos ingresados correspondan  con la definición del diccionario de datos \ref{dd:Evidencia}. \Trayref{E}
		\UCpaso Obtiene la fecha del sistema para la evidencia y registra la evidencia.
		\UCpaso Muestra el mensaje (MSG-c7) de operación exitosa.\ref{MSGc7}
		\
	\end{UCtrayectoria}
		
	\begin{UCtrayectoriaA}{A}{Cancelar operación}{El usuario abandona el Caso de Uso.}
			\UCpaso[\UCactor] Decide ya no registrar la evidencia.
			\UCpaso[\UCactor] Oprime el botón \IUbutton{Cancelar}.
			\UCpaso Actualiza la pantalla removiendo el campo de atención y no la registra.
	\end{UCtrayectoriaA}
